\documentclass[12pt]{article}

\usepackage{graphicx}
\usepackage{paralist}
\usepackage{amsfonts}

\oddsidemargin 0mm
\evensidemargin 0mm
\textwidth 150mm
\textheight 200mm
\renewcommand\baselinestretch{1.0}

\pagestyle {plain}
\pagenumbering{arabic}

\newcounter{stepnum}

\title{Assignment 5, Part 1, Specification}
\author{SFWR ENG 2AA4}

\begin {document}

\maketitle

The purpose of this software design exercise is to design, specify, implement and test a module for storing the state
of an Freecell game.
\newpage 
\section* {CardADT Module}

\subsection*{Module}
CardT
\subsection* {Uses}

N/A

\subsection* {Syntax}

\subsubsection* {Exported Constants}
Size = 52 \\
\subsubsection* {Exported Types}
	CardT = ?
\subsubsection* {Exported Access Programs}

\begin{tabular}{| l | l | l | p{7cm} |}
\hline
\textbf{Routine name} & \textbf{In} & \textbf{Out} & \textbf{Exceptions}\\

\hline
CardT & string, string & CardT & \\
\hline
Suit & ~ & string & \\
\hline
Face & ~ & string & \\
\hline



\end{tabular}

\subsection* {Semantics}

\subsubsection* {State Variables}

$face$: string\\
$suit$: string\\


\subsubsection* {State Invariant}

None

\subsubsection* {Assumptions}

The CardT method is called for the abstract object before any other access routine is called for that
object.  The CardT method can be used to return the state of the game to the state of a new game.

\subsubsection* {Access Routine Semantics}

CardT(a, b):
\begin{itemize}
\item transition: 
 face, suit = a,b
\item output:
 $out := \mathit{self}$
\item exception :
none

\end{itemize}

Suit():
\begin{itemize}
\item output: 
$out :=$  suit
\item exception :
none
\end{itemize}

Face():
\begin{itemize}
	\item output: 
	$out :=$ face
	\item exception :
	none
\end{itemize}


\newpage
\section* {DeckOfCardsADT Module}

\subsection*{Module}
DeckOfCardsT
\subsection* {Uses}
CardT

\subsection* {Syntax}

\subsubsection* {Exported Constants}
Size = 52 \\
\subsubsection* {Exported Types}
DeckofCardsT = ?
\subsubsection* {Exported Access Programs}

\begin{tabular}{| l | l | l | p{7cm} |}
	\hline
	\textbf{Routine name} & \textbf{In} & \textbf{Out} & \textbf{Exceptions}\\
	\hline
	DeckOfCardsT &  & CardT & \\
	\hline
	getColor& CardT & string & \\
	\hline
	shuffle & ~ & CardT & \\
	\hline
	dealCard & ~ & CardT & \\
	\hline

\end{tabular}

\subsection* {Semantics}

\subsubsection* {State Variables}

$deck : cardT$

\subsubsection* {State Invariant}

None

\subsubsection* {Assumptions}

The DeckOfCardsT method is called for the abstract object before any other access routine is called for that
object.  The DeckOfCardsT method can be used to return the state of the game to the state of a new game.

\subsubsection* {Access Routine Semantics}

DeckOfCardsT():
\begin{itemize}
	\item transition: 
	string faces[ ] = \{"Ace", "2", "3", "4", "5", "6", "7", "8", "9", "10", "Jack", "Queen", "King"\}\\
	string suits[ ] = \{"Hearts", "Diamonds", "Clubs", "Spades"\}
	\item output:
	$out := \mathit{deck}$
	\item exception :
	none

\end{itemize}

getColor(a):
\begin{itemize}
	\item output: 
	$out :=$ color
	\item exception :
	none
\end{itemize}

shuffle():
\begin{itemize}
	\item output: 
	$out :=$ face, suit
	\item exception
	None
\end{itemize}

dealCard():
\begin{itemize}
	\item output: 
	$out := deck$
	\item exception
	None
\end{itemize}


\newpage
\section* {BoardADT Module}

\subsection*{Module}
BoardT 
\subsection* {Uses}
CardT, DeckOfCardsT

\subsection* {Syntax}

\subsubsection* {Exported Constants}
Size = 52 \\
\subsubsection* {Exported Types}
BoardT = ?
\subsubsection* {Exported Access Programs}

\begin{tabular}{| l | l | l | p{7cm} |}
	\hline
	\textbf{Routine name} & \textbf{In} & \textbf{Out} & \textbf{Exceptions}\\
	
	\hline
	BoardT &  & CardT, BoardT, BoardT  & \\
	\hline
	get & CardT & BoardT & \\
	\hline
	move & CardT,CardT & BoardT & \\
	\hline
	put & CardT & BoardT & \\
	\hline
	put & CardT,CardT & BoardT & \\
	\hline
	check & CardT  & BoardT \\
	\hline
	
\end{tabular}

\subsection* {Semantics}

\subsubsection* {State Variables}

$s : CardT$\\
$a : BoardT$\\
$b : BoardT$

\subsubsection* {State Invariant}

None

\subsubsection* {Assumptions}

The BoardT method is called for the abstract object before any other access routine is called for that
object.  The BoardT method can be used to return the state of the game to the state of a new game.

\subsubsection* {Access Routine Semantics}

BoardT():
\begin{itemize}
	\item transition:\\ 
	s := CardT S[7][8] \\
	a := BoardT D[1][4] \\
	b := BoardT D[1][4]
	
	\item output:
	$out := \mathit{self}$
	\item exception :
	none
	
\end{itemize}

get(c):
\begin{itemize}
	\item transition:\\ 
	s[0][0]:= c
	\item output: 
	$out :=$ a
	\item exception:\\
	s[i][j] $\neq$ null $\Rightarrow$ invalid\_argument
\end{itemize}

move(a, b):
\begin{itemize}
	\item transition:
	(getColor(a)$\neq$getColor(b) $\land$ $|a.Face()-b.Face()|=1$) $\Rightarrow$ put(a, b)
	
	\item output: 
	$out := s,a,b$
	\item exception:
	none
\end{itemize}

put(a, b):
\begin{itemize}
	\item transition:
	s[i][j] = a, s[i+1][j] = b
	
	\item output: 
	$out := s,a, b$
	\item exception:
	(i = 6) $\Rightarrow$ Invalid\_moving
\end{itemize}

put(a):
\begin{itemize}
	\item transition:
	$(a.Face() \neq Ace \land b[0][j]=null) \Rightarrow$ j $\in$ \{0..3\} b[0][j] = a\\
	$(a.Face() = Ace \land b[0][j]=null) \Rightarrow$ j $\in$ \{0..3\} a[0][j] = a
	\item output: 
	$out := b, a $
	\item exception:
	none
\end{itemize}

check(a):
\begin{itemize}
	\item transition:
	$(a.Suit() = b[i][j].Suit() \land |a.Face()-b[i][j].Face()|=1) \Rightarrow$ b[i+1][j]=a
	\item output: 
	$out := b$
	\item exception:
	none
\end{itemize}

\end {document}

